%%%%%%%%%%%%%%%%%%%%%%%%%%%%%%%%%%%%%%% 
% Deedy CV/Resume
% XeLaTeX Template
% Version 1.0 (5/5/2014)
% 
% This template has been downloaded from:
% http://www.LaTeXTemplates.com
% 
% Original author:
% Debarghya Das (http://www.debarghyadas.com)
% With extensive modifications by:
% Vel (vel@latextemplates.com)
% 
% License:
% CC BY-NC-SA 3.0 (http://creativecommons.org/licenses/by-nc-sa/3.0/)
% 
% Important notes:
% This template needs to be compiled with XeLaTeX.
% 
%%%%%%%%%%%%%%%%%%%%%%%%%%%%%%%%%%%%%% 

\documentclass[a4paper]{deedy-resume} % Use US Letter paper, change to a4paper for A4 

\newcommand{\onepage}[1]{
  \ifdefined\onep
  #1
  \fi
}

\begin{document}

\namesection{Vaibhav}{Agarwal}{
  \urlstyle{same}\url{http://cse.iitkgp.ac.in/~vaibhavagarwal}
  \href{mailto:vaiagarwal96@iitkgp.ac.in}{ vaiagarwal96@iitkgp.ac.in}\\
  Reach me at : Meghnad Saha Hall, Indian Institute of Technology Kharagpur - 721302
}

\begin{minipage}[t]{0.33\textwidth} % The left column takes up 33% of the text width of the page

  \section{Education} 

  \descriptbig{IIT Kharagpur}

  \descript{B.Tech in Computer Science and Engineering}
  \location{2014-Present | CGPA: \textit{8.6/10.0}}

  \sectionspace
  \sectionspace
  
  \descriptbig{Mahatma Hansraj Modern School, Jhansi (CBSE)}
  \descript{Senior Secondary Grade 12}
  \location{Overall: \textbf{95.4\%}}

  \sectionspace
  \sectionspace
  
  \descriptbig{Christ the King College, Jhansi (ICSE)}
  \descript{Secondary Grade 10}
  \location{Overall: \textbf{94\%}}    
    
  \sectionspace

  \section{Links} 

  Github:// \href{https://github.com/VaibhavAgarwalVA}{\bf VaibhavAgarwalVA} \\
  LinkedIn:// \href{https://www.linkedin.com/in/VaibhavAgarwalVA}{\bf VaibhavAgarwalVA} \\
  CodeChef:// \href{https://www.codechef.com/users/infiniticoder}{\bf infiniticoder} \\
  
  \sectionspace

  \section{CS Coursework}
  Algorithms 1, Algorithms 2 *\\
  Software Engineering \\
  Computer Organization *\\
  Programming and Data Strcutures\\
  Compilers *, Automata Theory\\
  Discrete Structures\\
  Linear Algebra *, Graph Theory *\\
  Probability and Statistics\\
  Switching Circuit and Logic \\

  {\footnotesize \textit{\textbf{(* denotes ongoing courses) }}} \\

  \sectionspace

  \section{Skills}

  \runsubsection{}
  \descriptbig{Programming Languages}
  C/C++ \textbullet{} Java \textbullet{} Python \textbullet{} C\#
  \textbullet{} php 

  \sectionspace
  \sectionspace

  \descriptbig{Operating Systems}
  Ubuntu \textbullet{} Microsoft Windows

  \sectionspace
  \sectionspace

  \descriptbig{Utilities}
  Git \textbullet{} Linux Shell \textbullet{} OpenCV \textbullet{} GDB \textbullet{}
  \LaTeX \textbullet{} Docker

  \sectionspace
  \sectionspace

  \section{Interests}
  Algorithms and Graph Theory \\
  Robotics \\
  Data Analytics \\
  Artificial Intelligence \\
  Competitive Programming \\
  Software Testing

\end{minipage}
\hfill
\begin{minipage}[t]{0.66\textwidth}

  \section{Awards and Achievements}

\fontspec[Path = fonts/raleway/]{Raleway-Light}\fontsize{11pt}{11pt}\selectfont {

  \begin{tabular}{rll}
    2014	 & \bold{All India Rank 217}, JEE Advanced, among 150,000 candidates\\
    2014	 & \bold{All India Percentile 99.99}, JEE Mains, among 1.5 million candidates\\
    2016     & \bold{RoboCup 2016, Germany} Represented IIT Kharagpur\\
    2015     & \bold{ACM ICPC Asia Amritapuri Regionals} finalist\\
    2014	 & \bold{Academic Achievement Award} IIT Kharagpur\\
    2015     & \bold{FIRA 2015, South Korea} Bronze Medal for IIT Kharagpur\\
    2013     & \bold{KVPY} (Govt. of India) scholarship awardee\\
    2010     & \bold{National Talent Search} (Govt of India) scholarship awardee\\
  \end{tabular}
}\\
\normalfont

  \sectionspace


  \section{Experience and Projects}

  \runsubsection{}
  \descriptnonewline{Undergraduate Researcher}\location{| May'16-July'16}
  \location{Indian Institute of Technology Kharagpur}
  \vspace{\topsep}
  \begin{tightitemize}
  \item Developed positioning module for humanoid robots using Voronoi cell method and dynamic role assignment using maximum bipartite matching Hungarian algorithm.
  \item	Integrated successfully. Increased time efficiency by 30\%.
  \item Participated in RoboCup 2016, Germany.
  \end{tightitemize}

  \sectionspace

  \runsubsection{}
  \descriptnonewline{Undergraduate Researcher}\location{| May'15-July'15}
  \location{Indian Institute of Technology Kharagpur}
  \begin{tightitemize}
  \item Created a path planner for differential drive autonomous robots.
  \item	Decreased slip parameter to less than 1\%.
  \item The team won Bronze at FIRA 2015, South Korea.
  \end{tightitemize}

  \sectionspace

  \runsubsection{}
  \descriptnonewline{Software Developer}\location{Meghnad Saha Hall | Jan'16 - Present}
  \begin{tightitemize}
  \item Created a software to extract details from a plot and tabulate them using OpenCV libraries and tessaract.
  \end{tightitemize}

  \sectionspace

  \runsubsection{}
  \descriptnonewline{Back End Developer}\location{Game Thoery Society,
    IIT Kharagpur}
  \begin{tightitemize}
  \item Contributed to back end of an online game made using CodeIgnitor framework in php.
  \item Played by over 1000 participants from across India.
  \end{tightitemize}


  \sectionspace

  \runsubsection{}
  \descriptnonewline{Image Processing Mentor}\location{IIT Kharagpur | Dec'15}
  \begin{tightitemize}
  \item Mentored over 50 students in IEEE certified workshop.
  \item Taught various image processing techniques and algorithms.
  \end{tightitemize}

  \sectionspace

  \runsubsection{}
  \descriptnonewline{Code Club}\location{Programming Mentor | IIT Kharagpur}
  \begin{tightitemize}
  \item Problem setter and tester in coding contests at IIT Kharagpur.
  \item Mentored over 100 students in algorithms and data structures.
  \item Provided logistics support for Bitwise 2016, which saw the participation of over 1000 programmers across the globe.
  \end{tightitemize}
  
  \sectionspace

  \runsubsection{}
  \descriptnonewline{Grid Follower}\location{Robotics Workshop, IIT Kharagpur | Dec'14}
  \begin{tightitemize}
  \item Attended Texas Instruments certified workshop and built an autonomous grid follower robot.
  \item Programmed the algorithm using DFS traversal to cover the entire grid in Atmel Studio.
  \end{tightitemize}

  \section{Miscellaneous}
  \vspace{\topsep}
  \begin{tightitemize}
  \item Head of Algorithmic Game Society, IIT Kharagpur.
  \item Attended RoboCup Symposium at Leipzig, Germany.
  \item Occasional contributor to \textb{Open Source}
    (\href{https://github.com/VaibhavAgarwalVA}{my Github account})
  \item Sub Head of Maths Olympiad Team at MS Hall, IIT Kharagpur.
  \item Winner at several programming contests.
  \item Sporadic blogger at my personal blog  (\href{https://iitiantheoryoflife.blogspot.in}{\textit{An IITian's theory of life.}}).
  \end{tightitemize}

\end{minipage}

\end{document}
%%% Local Variables:
%%% mode: latex
%%% TeX-master: t
%%% TeX-engine: xetex
%%% End:
